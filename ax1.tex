\documentclass[fleqn]{report} 
\usepackage{amsmath, amsfonts}
\begin{document}
\title{Assignment 1 - Adaptive Quadrature}
\author{Ross Meikleham 1107023m}

Simpson's and composite Simpson's rules

\section{Question 1:}

Consider the Taylor expansion about the mid point $c = (a+b)/2$. 
Let $h =b-a$, then $a =c$$-h/2$ and $b=c+h/2$.

\begin{equation}
\begin{split}
f(a) & =f(c - \frac{h}{2})\\
& = f(c) - \frac{h}{2}f'(c) + \frac{h^2}{8}f''(c) - \frac{h^3}{48}f'''(c) + \frac{h^4}{384}f^{(iv)}(c) + O(h^5)
\end{split}
\end{equation}

\begin{equation}
\begin{split}
f(b) & =f(c + \frac{h}{2})\\
& = f(c) + \frac{h}{2}f'(c) + \frac{h^2}{8}f''(c) + \frac{h^3}{48}f'''(c) + \frac{h^4}{384}f^{(iv)}(c) + O(h^5)
\end{split}
\end{equation}

\begin{equation}
\begin{split}
f(x) & =f(c + (x - c))\\
& = f(c) + (x - c)f'(c) + \frac{(x-c)^2}{2!}f''(c) + \frac{(x-c)^4}{4!}f^{(iv)}(c) + O(h^5) 
\end{split}
\end{equation}
\\
Substitute $x=c+uh/2$ into (3) and integrate (3) term by term, then

\begin{equation}
M(f) = \int_a^b \! f(x) \, \mathrm{d}x. 
= hf(c) + \frac{h^3}{24}f''(c) + \frac{h^5}{1920}f^{(iv)}(c) + O(h^7)
\end{equation}
\\
Let N(f,h) be the discretization method that approximates M(f). In this case it will be Simpson's rule:
\begin{equation}
N(f,h) = \frac{h}{6}(f(a) + 4f(c) + f(b))
\end{equation}
\\
Substitute (1) and (2) into N(f,h):
\begin{equation}
\begin{split}
N(f,h) & =\frac{h}{6}[6f(c) + \frac{h^2}{4}f''(c) + \frac{h^4}{192}f^{(iv)}(c) + O(h^6)]\\
& = hf(c) + \frac{h^3}{24}f''(c) + \frac{h^5}{1152}f^{(iv)}(c) + O(h^7)
\end{split}
\end{equation}


We can now calculate the Error E(h):
\begin{equation}
\begin{split}
E(h) & =M(f) - N(f,h)\\
& = \frac{h^5}{1920}f^{(iv)}(c) -  \frac{h^5}{1152}f^{(iv)}(c) + O(h^7) \\
   % \text{Can discard O(h^7) as 5 < 7}\\
& = h^5f^{(iv)}(c)(\frac{1}{1920} - \frac{1}{1152})\\
& = \frac{-h^5f^{(iv)}(c)}{2880}\\
& = \frac{-(b-a)^5f^{(iv)}(c)}{2880}
\end{split}
\end{equation}
\\
as required.

\section{Question 2:}
We'll generate N equal sub-intervals of [a,b], for some N $\in \mathbb{N}$\\
Label the sub-intervals as $I_{i}$ with end-points $[x_{2i-2}, x_{2i}]$, for i = 1,...,N\\
The intervals are of equal length $x_{2i} - x_{2i-2} = 2h = \frac{(b - a)}{N}$\\
Where there are 2N + 1 quadrature points $x_{i} = a + ih,$ i = 0,...,2N
\\
Then the simpson rule on $I_{i}$ is:
\begin{equation}
\begin{split}
\int_{I_i} \! f(x) \, \mathrm{d}x. & = \frac{2h}{6}
(f(x_{2i-2}) + 4f(\frac{x_{2i-2} + x_{2i}}{2}) + f(x_{2i}))\\
& = \frac{h}{3}(f(x_{2i-2}) + 4f(x_{2i-1}) + f(x_{2i}))
\end{split}
\end{equation}
\\
and the sum of integrals over the sub-intervals is:

\begin{equation}
\begin{split}
\int_a^b \! f(x) \,\mathrm{d}x. & \approx  \sum_{i=1}^{N} \int_{I_i} \! f(x) \,\\
& = \sum_{i=1}^{N} \frac{h}{3}(f(x_{2i-2}) + 4f(x_{2i-1}) + f(x_{2i}))\\
& = \frac{h}{3} \sum_{i=1}^{N} (f(x_{2i-2}) + 4f(x_{2i-1}) + f(x_{2i}))\\
& = \frac{h}{3} (f(x_0) + f(x_{2N}) +  2\sum_{i=1}^{N-1} f(x_{2i}) + 4\sum_{i=1}^{N} f(x_{2i-1}))
\end{split}
\end{equation}    
\\
Question 3:\\

Note that:
\begin{equation}
\begin{split}
Sc(f;a,b,1) &= \frac{h}{3} (f(x_0) + f(x_{2}) +  2\sum_{i=1}^{0} f(x_{2i}) + 4\sum_{i=1}^{1} f(x_{2i-1}))\\
            & = \frac{h}{3} (f(x_0) + 4f(x_1) + f(x_2))\\
            & = \frac{h}{3} (f(a) + 4f(\frac{a + b}{2}) + f(b))\\
            & = S(f;a,b)\\
\end{split}
\end{equation}
\begin{equation}
\implies Ec(f;a,b,1) = E(f;a,b)
\end{equation}
\\
As $f^{iv}$ is constant: $f^{(iv)}(c) = \alpha$ for $ \forall c \in [a,b]$ 
where $\alpha$ is a constant\\
Then:
\begin{equation}
E(f;a,b) = -\frac{(b - a)^5}{2880}\alpha
\end{equation}

\begin{equation}
E_c(f;a,b,2) = -\frac{(b - a)^5}{2880 * 2^4}\alpha
\end{equation}

\begin{equation}
\implies E(f;a,b) = 16E_c(f;a,b,2)
\end{equation}
\\
Let $M(f) = \int_a^b \! f(x) \, \mathrm{d}x$ \\
From definition of Numerical Schemes
\begin{equation}
\begin{split}
&M(f) = S_c(f;a,b,2) + E_c(f;a,b,2)\\
&M(f) = S_c(f;a,b,1) + E_c(f;a,b,1)
\end{split}
\end{equation}
\\
Then:
\begin{equation}
\begin{split}
S_c(f;a,b,2) - S_c(f;a,b,1) & = M(f) - E_c(f;a,b,2) - M(f) + E_c(f;a,b,1)\\
& = E_c(f;a,b,1) - E_c(f;a,b,2)\\
& = E(f;a,b) - E_c(f;a,b,2)\\
& = 16E_c(f;a,b,2) - E_c(f;a,b,2)\\
& = 15E_c(f;a,b,2)\\
\end{split}
\end{equation}
\begin{equation}
\iff \frac{1}{15}(S_c(f;a,b,2) - S_c(f;a,b,1)) = E_c(f;a,b,2)
\end{equation}
\\ as required.\\\\
Let $E(f;a,b,2) = E_c(f;a,b,2)$ and by $M(f) = S_c(f;a,b,2) + E_c(f;a,b,2)$ 
it follows that:\\
\begin{equation}
\int_a^b \! f(x) \, \mathrm{d}x  = S_c(f;a,b,2) + E(f;a,b,2)
\end{equation}

\end{document} 



