\documentclass[fleqn]{report} 
\usepackage{amsmath, amsfonts}
\begin{document}
\title{Assignment 1 - Adaptive Quadrature}
\author{Ross Meikleham 1107023m}

Simpson's and composite Simpson's rules

\section{Question 1:}

Consider the Taylor expansion about the mid point $c = (a+b)/2$. 
Let $h =b-a$, then $a =c$$-h/2$ and $b=c+h/2$.

\begin{equation}
\begin{split}
f(a) & =f(c - \frac{h}{2})\\
& = f(c) - \frac{h}{2}f'(c) + \frac{h^2}{8}f''(c) - \frac{h^3}{48}f'''(c) + \frac{h^4}{384}f^{(iv)}(c) + O(h^5)
\end{split}
\end{equation}

\begin{equation}
\begin{split}
f(b) & =f(c + \frac{h}{2})\\
& = f(c) + \frac{h}{2}f'(c) + \frac{h^2}{8}f''(c) + \frac{h^3}{48}f'''(c) + \frac{h^4}{384}f^{(iv)}(c) + O(h^5)
\end{split}
\end{equation}

\begin{equation}
\begin{split}
f(x) & =f(c + (x - c))\\
& = f(c) + (x - c)f'(c) + \frac{(x-c)^2}{2!}f''(c) + \frac{(x-c)^4}{4!}f^{(iv)}(c) + O(h^5) 
\end{split}
\end{equation}
\\
Substitute $x=c+uh/2$ into (3) and integrate (3) term by term, then

\begin{equation}
M(f) = \int_a^b \! f(x) \, \mathrm{d}x. 
= hf(c) + \frac{h^3}{24}f''(c) + \frac{h^5}{1920}f^{(iv)}(c) + O(h^7)
\end{equation}
\\
Let N(f,h) be the discretization method that approximates M(f). In this case it will be Simpson's rule:
\begin{equation}
N(f,h) = \frac{h}{6}(f(a) + 4f(c) + f(b))
\end{equation}
\\
Substitute (1) and (2) into N(f,h):
\begin{equation}
\begin{split}
N(f,h) & =\frac{h}{6}[6f(c) + \frac{h^2}{4}f''(c) + \frac{h^4}{192}f^{(iv)}(c) + O(h^6)]\\
& = hf(c) + \frac{h^3}{24}f''(c) + \frac{h^5}{1152}f^{(iv)}(c) + O(h^7)
\end{split}
\end{equation}


We can now calculate the Error E(h):
\begin{equation}
\begin{split}
E(h) & =M(f) - N(f,h)\\
& = \frac{h^5}{1920}f^{(iv)}(c) -  \frac{h^5}{1152}f^{(iv)}(c) + O(h^7) \\
   % \text{Can discard O(h^7) as 5 < 7}\\
& = h^5f^{(iv)}(c)(\frac{1}{1920} - \frac{1}{1152})\\
& = \frac{-h^5f^{(iv)}(c)}{2880}\\
& = \frac{-(b-a)^5f^{(iv)}(c)}{2880}
\end{split}
\end{equation}
\\
as required.

\section{Question 2:}
We'll generate N/2 equal sub-intervals of [a,b], for some even N $\in \mathbb{N}$\\
Let M = N/2\\
Label the sub-intervals as $I_{i}$ with end-points $[x_{2i-2}, x_{2i}]$, for i = 1,...,M\\
The intervals are of equal length $x_{2i} - x_{2i-2} = 2h = \frac{(b - a)}{M}$\\
Where there are N + 1 quadrature points $x_{i} = a + ih,$ i = 0,...,N
\\
Then the simpson rule on $I_{i}$ is:
\begin{equation}
\begin{split}
\int_{I_i} \! f(x) \, \mathrm{d}x. & = \frac{2h}{6}
(f(x_{2i-2}) + 4f(\frac{x_{2i-2} + x_{2i}}{2}) + f(x_{2i}))\\
& = \frac{h}{3}(f(x_{2i-2}) + 4f(x_{2i-1}) + f(x_{2i}))
\end{split}
\end{equation}
\\
and the sum of integrals over the sub-intervals is:

\begin{equation}
\begin{split}
\int_a^b \! f(x) \,\mathrm{d}x. & \approx  \sum_{i=1}^{M} \int_{I_i} \! f(x) \,\\
& = \sum_{i=1}^{M} \frac{h}{3}(f(x_{2i-2}) + 4f(x_{2i-1}) + f(x_{2i}))\\
& = \frac{h}{3} \sum_{i=1}^{M} (f(x_{2i-2}) + 4f(x_{2i-1}) + f(x_{2i}))\\
& = \frac{h}{3} (f(x_0) + f(x_N) +  2\sum_{i=1}^{M} f(x_{2i}) + 4\sum_{i=2}^{M} f(x_{2i}))
\end{split}
\end{equation}    



Question 3:

\end{document} 
